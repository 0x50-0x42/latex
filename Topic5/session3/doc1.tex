\documentclass{article}

\usepackage{hyperref}
\usepackage{lipsum}


\begin{document}
	\section{Overview}
	\paragraph{}
	
	Section \ref{subsec:def} covers some notations used in this article. Section \ref{subsec:who} addresses which users should read it. The section, \nameref{subsec:reqs} on page number \pageref{subsec:reqs} points to what the users need. Finally the table \ref{tbl:items} on page number \pageref{tbl:items} lists out the different items available in the system.
	
	\section{Introduction}
	\label{sec:intro}
	\paragraph{}
	
	\lipsum[1-2]
	
	\subsection{Definition}
	\label{subsec:def}
	\paragraph{}
	
	\lipsum[1-2]
	
	\subsection{Audience}
	\label{subsec:who}
	\paragraph{}
	
	\lipsum[1-2]
	
	\begin{table}[h]
		\centering
		\begin{tabular}{cc}
			\hline
			No. & Item\\
			\hline
			1 & Pen\\
			\hline
			2 & Pencil\\
			\hline
			3 & Scale\\
			\hline
		\end{tabular}
	
	\caption{Items}
	\label{tbl:items}
	\end{table}

\section{Specifications}
\label{sec:specs}
\paragraph{}

\lipsum[1-2]

\subsection{Requirements}
\label{subsec:reqs}
\paragraph{}

\lipsum[1-2]

\subsection{Functional}
\label{subsec:func}
\paragraph{}

\lipsum[1-2]
\end{document}