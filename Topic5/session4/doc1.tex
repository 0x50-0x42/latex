\documentclass[12pt]{article}

\usepackage[margin=1in]{geometry}

\usepackage{imakeidx}
\makeindex[columns=1, title=Back Index, intoc]

\title{Tiger}
\author{Wiki}
\date{}

\begin{document}
	\maketitle
	\tableofcontents
	\pagebreak
	
	\section{Introduction}
	\paragraph{}
	The tiger \index{tiger} (Panthera tigris) \index{Panthera} is the largest species among the Felidae and classified in the genus Panthera. It is most recognisable for its dark vertical stripes on reddish-orange fur with a lighter underside. It is an apex predator, primarily preying on ungulates such as deer and wild boar. It is territorial and generally a solitary but social predator, requiring large contiguous areas of habitat, which support its requirements for prey and rearing of its offspring. Tiger cubs stay with their mother for about two years, before they become independent and leave their mother's home range to establish their own.
	
	\section{Habitat}
	\paragraph{}
	The tiger once ranged widely from the Eastern Anatolia Region in the West to the Amur River basin, and in the south from the foothills of the Himalayas to Bali in the Sunda islands. Since the early $20^{th}$ century, tiger populations have lost at least $93\%$ of their historic range and have been extirpated in Western and Central Asia\index{Asia|see{habitat}}, from the islands of Java and Bali, and in large areas of Southeast and South Asia and China,. Today's tiger range is fragmented, stretching from Siberian temperature forests to subtropical and tropical forests on the Indian\index{habitat!India} subcontinent and Sumatra. The tiger is listed as Endangered on the IUCN Red List since 1986. As of 2015, the global wild tiger population was estimated to number between $3,062$ and $3,948$ mature individuals, down from around $100,000$ at the start of the $20^{th}$ century, with most remaining populations occurring in small pockets isolated from each other. Major reasons for population decline include habitat destruction, habitat fragmentation and poaching. This, coupled with the fact that it lives in some of the more densely populated places on Earth, has caused significant conflict with humans.
	
	\pagebreak
	\section{More Information}
	\paragraph{}
	The tiger\index{tiger} is among the most recognizable and popular of the world's charismatic megafauna. It featured prominently in ancient mythology and folklore and continues to be depicted in modern films and literature, appearing on many flags, coats of arms and as mascots for sporting teams. The tiger is the national animal of India,\index{habitat!India} Bangladesh,\index{habitat!Bangladesh} Malaysia\index{habitat!Malaysia} and South Asia\index{habitat!South Korea}.
	
	
	
	\printindex % print the index
	
\end{document}