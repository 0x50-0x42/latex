\documentclass[11pt]{article}

\usepackage[margin=0.75in, paperwidth=8.5in, paperheight=11in]{geometry}

\parindent 0 px


\begin{document}
The function $f(x)=(x-3)^2+\frac{1}{2}$ has domain $\mathrm{D}_f:(-\infty, \infty)$ and range $\mathrm{R}_f:\left[\frac{1}{2}, \infty\right)$.\\

\section{Limits}

$\lim_{x \to a}$\\ % if we donot use \limits

$\lim\limits_{x \to a} f(x)$\\

$\lim\limits_{x \to a^-} f(x)$\\

$\displaystyle{\lim\limits_{x\to a}\frac{f(x)-f(a)}{x-a}=f'(a)}$\\

\section{Integrals}

$\int\sin x\,dx=-\cos x+C$\\

$\displaystyle{\int\sin x\,dx=-\cos x+C}$\\

$\displaystyle{\int_a^b}$\\

$\displaystyle{\int\limits_a^b}$\\

$\displaystyle{\int\limits_{a}^{b}x^2\,dx=\left[\frac{x^3}{3}\right]_{a}^{b}=\frac{b^3}{3}-\frac{a^3}{3}}$\\

\section{Summations}

$\sum$\\

$\displaystyle{\sum}$\\

$\displaystyle{\sum\limits_{n=1}^{\infty}ar^n=a+ar+ar^2+\cdots +ar^n}$\\

\section{Integrals, limits and summations--all together}

$\displaystyle{\int_a^b f(x)\,dx=\lim\limits_{x \to \infty}\sum\limits_{k=1}^{n}f(x_k)\cdot\Delta x}$\\

\section{Vectors}

$\vec{v}=v_1 \vec{i}+v_2 \vec{j}=\langle v_1, v_2 \rangle$
\end{document}